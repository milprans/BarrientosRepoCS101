% Options for packages loaded elsewhere
\PassOptionsToPackage{unicode}{hyperref}
\PassOptionsToPackage{hyphens}{url}
%
\documentclass[
]{article}
\usepackage{amsmath,amssymb}
\usepackage{iftex}
\ifPDFTeX
  \usepackage[T1]{fontenc}
  \usepackage[utf8]{inputenc}
  \usepackage{textcomp} % provide euro and other symbols
\else % if luatex or xetex
  \usepackage{unicode-math} % this also loads fontspec
  \defaultfontfeatures{Scale=MatchLowercase}
  \defaultfontfeatures[\rmfamily]{Ligatures=TeX,Scale=1}
\fi
\usepackage{lmodern}
\ifPDFTeX\else
  % xetex/luatex font selection
\fi
% Use upquote if available, for straight quotes in verbatim environments
\IfFileExists{upquote.sty}{\usepackage{upquote}}{}
\IfFileExists{microtype.sty}{% use microtype if available
  \usepackage[]{microtype}
  \UseMicrotypeSet[protrusion]{basicmath} % disable protrusion for tt fonts
}{}
\makeatletter
\@ifundefined{KOMAClassName}{% if non-KOMA class
  \IfFileExists{parskip.sty}{%
    \usepackage{parskip}
  }{% else
    \setlength{\parindent}{0pt}
    \setlength{\parskip}{6pt plus 2pt minus 1pt}}
}{% if KOMA class
  \KOMAoptions{parskip=half}}
\makeatother
\usepackage{xcolor}
\usepackage[margin=1in]{geometry}
\usepackage{color}
\usepackage{fancyvrb}
\newcommand{\VerbBar}{|}
\newcommand{\VERB}{\Verb[commandchars=\\\{\}]}
\DefineVerbatimEnvironment{Highlighting}{Verbatim}{commandchars=\\\{\}}
% Add ',fontsize=\small' for more characters per line
\usepackage{framed}
\definecolor{shadecolor}{RGB}{248,248,248}
\newenvironment{Shaded}{\begin{snugshade}}{\end{snugshade}}
\newcommand{\AlertTok}[1]{\textcolor[rgb]{0.94,0.16,0.16}{#1}}
\newcommand{\AnnotationTok}[1]{\textcolor[rgb]{0.56,0.35,0.01}{\textbf{\textit{#1}}}}
\newcommand{\AttributeTok}[1]{\textcolor[rgb]{0.13,0.29,0.53}{#1}}
\newcommand{\BaseNTok}[1]{\textcolor[rgb]{0.00,0.00,0.81}{#1}}
\newcommand{\BuiltInTok}[1]{#1}
\newcommand{\CharTok}[1]{\textcolor[rgb]{0.31,0.60,0.02}{#1}}
\newcommand{\CommentTok}[1]{\textcolor[rgb]{0.56,0.35,0.01}{\textit{#1}}}
\newcommand{\CommentVarTok}[1]{\textcolor[rgb]{0.56,0.35,0.01}{\textbf{\textit{#1}}}}
\newcommand{\ConstantTok}[1]{\textcolor[rgb]{0.56,0.35,0.01}{#1}}
\newcommand{\ControlFlowTok}[1]{\textcolor[rgb]{0.13,0.29,0.53}{\textbf{#1}}}
\newcommand{\DataTypeTok}[1]{\textcolor[rgb]{0.13,0.29,0.53}{#1}}
\newcommand{\DecValTok}[1]{\textcolor[rgb]{0.00,0.00,0.81}{#1}}
\newcommand{\DocumentationTok}[1]{\textcolor[rgb]{0.56,0.35,0.01}{\textbf{\textit{#1}}}}
\newcommand{\ErrorTok}[1]{\textcolor[rgb]{0.64,0.00,0.00}{\textbf{#1}}}
\newcommand{\ExtensionTok}[1]{#1}
\newcommand{\FloatTok}[1]{\textcolor[rgb]{0.00,0.00,0.81}{#1}}
\newcommand{\FunctionTok}[1]{\textcolor[rgb]{0.13,0.29,0.53}{\textbf{#1}}}
\newcommand{\ImportTok}[1]{#1}
\newcommand{\InformationTok}[1]{\textcolor[rgb]{0.56,0.35,0.01}{\textbf{\textit{#1}}}}
\newcommand{\KeywordTok}[1]{\textcolor[rgb]{0.13,0.29,0.53}{\textbf{#1}}}
\newcommand{\NormalTok}[1]{#1}
\newcommand{\OperatorTok}[1]{\textcolor[rgb]{0.81,0.36,0.00}{\textbf{#1}}}
\newcommand{\OtherTok}[1]{\textcolor[rgb]{0.56,0.35,0.01}{#1}}
\newcommand{\PreprocessorTok}[1]{\textcolor[rgb]{0.56,0.35,0.01}{\textit{#1}}}
\newcommand{\RegionMarkerTok}[1]{#1}
\newcommand{\SpecialCharTok}[1]{\textcolor[rgb]{0.81,0.36,0.00}{\textbf{#1}}}
\newcommand{\SpecialStringTok}[1]{\textcolor[rgb]{0.31,0.60,0.02}{#1}}
\newcommand{\StringTok}[1]{\textcolor[rgb]{0.31,0.60,0.02}{#1}}
\newcommand{\VariableTok}[1]{\textcolor[rgb]{0.00,0.00,0.00}{#1}}
\newcommand{\VerbatimStringTok}[1]{\textcolor[rgb]{0.31,0.60,0.02}{#1}}
\newcommand{\WarningTok}[1]{\textcolor[rgb]{0.56,0.35,0.01}{\textbf{\textit{#1}}}}
\usepackage{graphicx}
\makeatletter
\def\maxwidth{\ifdim\Gin@nat@width>\linewidth\linewidth\else\Gin@nat@width\fi}
\def\maxheight{\ifdim\Gin@nat@height>\textheight\textheight\else\Gin@nat@height\fi}
\makeatother
% Scale images if necessary, so that they will not overflow the page
% margins by default, and it is still possible to overwrite the defaults
% using explicit options in \includegraphics[width, height, ...]{}
\setkeys{Gin}{width=\maxwidth,height=\maxheight,keepaspectratio}
% Set default figure placement to htbp
\makeatletter
\def\fps@figure{htbp}
\makeatother
\setlength{\emergencystretch}{3em} % prevent overfull lines
\providecommand{\tightlist}{%
  \setlength{\itemsep}{0pt}\setlength{\parskip}{0pt}}
\setcounter{secnumdepth}{-\maxdimen} % remove section numbering
\ifLuaTeX
  \usepackage{selnolig}  % disable illegal ligatures
\fi
\usepackage{bookmark}
\IfFileExists{xurl.sty}{\usepackage{xurl}}{} % add URL line breaks if available
\urlstyle{same}
\hypersetup{
  pdftitle={Rworksheet\_Barrientos\#4B},
  pdfauthor={Barrientos, Milfrance D.},
  hidelinks,
  pdfcreator={LaTeX via pandoc}}

\title{Rworksheet\_Barrientos\#4B}
\author{Barrientos, Milfrance D.}
\date{10/30/2024}

\begin{document}
\maketitle

\section{Using For Loop Function}\label{using-for-loop-function}

\begin{Shaded}
\begin{Highlighting}[]
\CommentTok{\#1. Using the for loop, create an R script that will display a 5x5 matrix as shown in}
\CommentTok{\# Figure 1. It must contain vectorA = [1,2,3,4,5] and a 5 x 5 zero matrix.}
\CommentTok{\# Hint Use abs() function to get the absolute value}

\NormalTok{vectorA }\OtherTok{\textless{}{-}} \FunctionTok{c}\NormalTok{(}\DecValTok{1}\NormalTok{,}\DecValTok{2}\NormalTok{,}\DecValTok{3}\NormalTok{,}\DecValTok{4}\NormalTok{,}\DecValTok{5}\NormalTok{)}
\NormalTok{matrixA }\OtherTok{\textless{}{-}} \FunctionTok{matrix}\NormalTok{(}\DecValTok{0}\NormalTok{,}\DecValTok{5}\NormalTok{,}\DecValTok{5}\NormalTok{)}

\ControlFlowTok{for}\NormalTok{ (i }\ControlFlowTok{in} \DecValTok{1}\SpecialCharTok{:}\DecValTok{5}\NormalTok{) \{}
  \ControlFlowTok{for}\NormalTok{ (j }\ControlFlowTok{in} \DecValTok{1}\SpecialCharTok{:}\DecValTok{5}\NormalTok{) \{}
\NormalTok{    matrixA[i,j] }\OtherTok{\textless{}{-}} \FunctionTok{abs}\NormalTok{(vectorA[i] }\SpecialCharTok{{-}}\NormalTok{ vectorA[j])}
\NormalTok{  \}}
\NormalTok{\}}
\FunctionTok{print}\NormalTok{(matrixA)}
\end{Highlighting}
\end{Shaded}

\begin{verbatim}
##      [,1] [,2] [,3] [,4] [,5]
## [1,]    0    1    2    3    4
## [2,]    1    0    1    2    3
## [3,]    2    1    0    1    2
## [4,]    3    2    1    0    1
## [5,]    4    3    2    1    0
\end{verbatim}

\begin{Shaded}
\begin{Highlighting}[]
\CommentTok{\# 2. Print the string "*" using for() function. The output should be the same as shown in Figure.}

\NormalTok{rows }\OtherTok{\textless{}{-}} \DecValTok{5} 
\ControlFlowTok{for}\NormalTok{ (i }\ControlFlowTok{in} \DecValTok{1}\SpecialCharTok{:}\NormalTok{rows) \{}
\FunctionTok{cat}\NormalTok{(}\FunctionTok{paste}\NormalTok{(}\FunctionTok{rep}\NormalTok{(}\StringTok{"*"}\NormalTok{, i), }\AttributeTok{collapse=}\StringTok{" "}\NormalTok{), }\StringTok{"}\SpecialCharTok{\textbackslash{}n}\StringTok{"}\NormalTok{)}
\NormalTok{\}}
\end{Highlighting}
\end{Shaded}

\begin{verbatim}
## * 
## * * 
## * * * 
## * * * * 
## * * * * *
\end{verbatim}

\begin{Shaded}
\begin{Highlighting}[]
\CommentTok{\#3. Get an input from the user to print the Fibonacci sequence starting from the 1st input}
\CommentTok{\# up to 500. Use repeat and break statements. Write the R Scripts and its output.}

\NormalTok{n }\OtherTok{\textless{}{-}} \DecValTok{600}
\NormalTok{n1 }\OtherTok{\textless{}{-}} \DecValTok{0}
\NormalTok{n2 }\OtherTok{\textless{}{-}} \DecValTok{1}
\ControlFlowTok{repeat}\NormalTok{ \{}
  
  \ControlFlowTok{if}\NormalTok{ (n1 }\SpecialCharTok{\textgreater{}=}\NormalTok{ n) \{}
    \FunctionTok{cat}\NormalTok{(n1, }\StringTok{"}\SpecialCharTok{\textbackslash{}n}\StringTok{"}\NormalTok{)}
\NormalTok{  \}}

  
  \ControlFlowTok{if}\NormalTok{ (n1 }\SpecialCharTok{\textgreater{}} \DecValTok{500}\NormalTok{) \{}
    \ControlFlowTok{break}
\NormalTok{  \}}
  
  
\NormalTok{  fib }\OtherTok{\textless{}{-}}\NormalTok{ n1 }\SpecialCharTok{+}\NormalTok{ n2}
\NormalTok{  n1 }\OtherTok{\textless{}{-}}\NormalTok{ n2}
\NormalTok{  n2 }\OtherTok{\textless{}{-}}\NormalTok{ fib}
\NormalTok{\}}
\end{Highlighting}
\end{Shaded}

\begin{verbatim}
## 610
\end{verbatim}

\#Using Basic Graphics (plot(),barplot(),pie(),hist())

\begin{Shaded}
\begin{Highlighting}[]
\NormalTok{household\_data }\OtherTok{\textless{}{-}} \FunctionTok{data.frame}\NormalTok{(}
  \AttributeTok{Shoe\_size =} \FunctionTok{c}\NormalTok{(}\FloatTok{6.5}\NormalTok{, }\FloatTok{9.0}\NormalTok{, }\FloatTok{8.5}\NormalTok{, }\FloatTok{8.5}\NormalTok{, }\FloatTok{10.5}\NormalTok{, }\FloatTok{7.0}\NormalTok{, }\FloatTok{9.5}\NormalTok{, }\FloatTok{9.0}\NormalTok{, }\FloatTok{13.0}\NormalTok{, }\FloatTok{7.5}\NormalTok{, }\FloatTok{10.5}\NormalTok{, }\FloatTok{8.5}\NormalTok{, }\FloatTok{12.0}\NormalTok{, }\FloatTok{10.5}\NormalTok{,}
\FloatTok{13.0}\NormalTok{, }\FloatTok{11.5}\NormalTok{, }\FloatTok{8.5}\NormalTok{, }\FloatTok{5.0}\NormalTok{, }\FloatTok{10.0}\NormalTok{, }\FloatTok{6.5}\NormalTok{, }\FloatTok{7.5}\NormalTok{, }
\FloatTok{8.5}\NormalTok{, }\FloatTok{10.5}\NormalTok{, }\FloatTok{8.5}\NormalTok{, }\FloatTok{10.5}\NormalTok{, }\FloatTok{11.0}\NormalTok{, }\FloatTok{9.0}\NormalTok{, }\FloatTok{13.0}\NormalTok{), }
  \AttributeTok{Height =} \FunctionTok{c}\NormalTok{(}\FloatTok{66.0}\NormalTok{, }\FloatTok{68.0}\NormalTok{, }\FloatTok{64.5}\NormalTok{, }\FloatTok{65.0}\NormalTok{, }\FloatTok{70.0}\NormalTok{, }\FloatTok{64.0}\NormalTok{, }\FloatTok{70.0}\NormalTok{, }\FloatTok{71.0}\NormalTok{, }\FloatTok{72.0}\NormalTok{, }\FloatTok{64.0}\NormalTok{, }\FloatTok{74.5}\NormalTok{, }\FloatTok{67.0}\NormalTok{, }\FloatTok{71.0}\NormalTok{, }\FloatTok{71.0}\NormalTok{, }
\FloatTok{77.0}\NormalTok{, }\FloatTok{72.0}\NormalTok{, }\FloatTok{59.0}\NormalTok{, }\FloatTok{62.0}\NormalTok{, }\FloatTok{72.0}\NormalTok{, }\FloatTok{66.0}\NormalTok{, }\FloatTok{64.0}\NormalTok{, }
\FloatTok{67.0}\NormalTok{, }\FloatTok{73.0}\NormalTok{, }\FloatTok{69.0}\NormalTok{, }\FloatTok{72.0}\NormalTok{, }\FloatTok{70.0}\NormalTok{, }\FloatTok{69.0}\NormalTok{, }\FloatTok{70.0}\NormalTok{),}
  \AttributeTok{Gender =} \FunctionTok{c}\NormalTok{(}\StringTok{"F"}\NormalTok{,}\StringTok{"F"}\NormalTok{,}\StringTok{"F"}\NormalTok{,}\StringTok{"F"}\NormalTok{,}\StringTok{"M"}\NormalTok{,}\StringTok{"F"}\NormalTok{,}\StringTok{"F"}\NormalTok{,}\StringTok{"F"}\NormalTok{,}\StringTok{"M"}\NormalTok{,}
\StringTok{"F"}\NormalTok{,}\StringTok{"M"}\NormalTok{,}\StringTok{"F"}\NormalTok{,}\StringTok{"M"}\NormalTok{,}\StringTok{"M"}\NormalTok{,}\StringTok{"M"}\NormalTok{,}\StringTok{"M"}\NormalTok{,}\StringTok{"F"}\NormalTok{,}\StringTok{"F"}\NormalTok{,}\StringTok{"M"}\NormalTok{,}
\StringTok{"F"}\NormalTok{,}\StringTok{"F"}\NormalTok{,}\StringTok{"M"}\NormalTok{,}\StringTok{"M"}\NormalTok{,}\StringTok{"F"}\NormalTok{,}\StringTok{"M"}\NormalTok{,}\StringTok{"M"}\NormalTok{,}\StringTok{"M"}\NormalTok{,}\StringTok{"M"}\NormalTok{)}
\NormalTok{)}
\NormalTok{household\_data}
\end{Highlighting}
\end{Shaded}

\begin{verbatim}
##    Shoe_size Height Gender
## 1        6.5   66.0      F
## 2        9.0   68.0      F
## 3        8.5   64.5      F
## 4        8.5   65.0      F
## 5       10.5   70.0      M
## 6        7.0   64.0      F
## 7        9.5   70.0      F
## 8        9.0   71.0      F
## 9       13.0   72.0      M
## 10       7.5   64.0      F
## 11      10.5   74.5      M
## 12       8.5   67.0      F
## 13      12.0   71.0      M
## 14      10.5   71.0      M
## 15      13.0   77.0      M
## 16      11.5   72.0      M
## 17       8.5   59.0      F
## 18       5.0   62.0      F
## 19      10.0   72.0      M
## 20       6.5   66.0      F
## 21       7.5   64.0      F
## 22       8.5   67.0      M
## 23      10.5   73.0      M
## 24       8.5   69.0      F
## 25      10.5   72.0      M
## 26      11.0   70.0      M
## 27       9.0   69.0      M
## 28      13.0   70.0      M
\end{verbatim}

\begin{Shaded}
\begin{Highlighting}[]
\CommentTok{\#a. What is the R script for importing an excel or a csv file? Display the first 6 rows of the dataset? Show your codes and its result}

\CommentTok{\#install.packages("readxl") for importing an excel file }
\CommentTok{\#install.packages("readr") for importing a csv file}
\FunctionTok{library}\NormalTok{(readxl)}
\FunctionTok{library}\NormalTok{(readr)}
\NormalTok{household\_data }\OtherTok{\textless{}{-}} \FunctionTok{read\_excel}\NormalTok{(}\StringTok{"C:/PROJ/household\_data.xlsx"}\NormalTok{)}

\FunctionTok{head}\NormalTok{(household\_data)}
\end{Highlighting}
\end{Shaded}

\begin{verbatim}
## # A tibble: 6 x 3
##   Shoe_size Height Gender
##       <dbl>  <dbl> <chr> 
## 1       6.5   66   F     
## 2       9     68   F     
## 3       8.5   64.5 F     
## 4       8.5   65   F     
## 5      10.5   70   M     
## 6       7     64   F
\end{verbatim}

\begin{Shaded}
\begin{Highlighting}[]
\CommentTok{\#b. Create a subset for gender(female and male). How many observations are there in Male? How about in Female? Write the R scripts and its output.}

\NormalTok{fem\_data }\OtherTok{\textless{}{-}} \FunctionTok{subset}\NormalTok{(household\_data, Gender }\SpecialCharTok{==} \StringTok{"F"}\NormalTok{)}
\NormalTok{male\_data }\OtherTok{\textless{}{-}} \FunctionTok{subset}\NormalTok{(household\_data, Gender }\SpecialCharTok{==} \StringTok{"M"}\NormalTok{)}

\NormalTok{num\_fem }\OtherTok{\textless{}{-}} \FunctionTok{nrow}\NormalTok{(fem\_data)}
\NormalTok{num\_male }\OtherTok{\textless{}{-}} \FunctionTok{nrow}\NormalTok{(male\_data)}

\FunctionTok{cat}\NormalTok{(}\StringTok{"Number of female observations: "}\NormalTok{, num\_fem, }\StringTok{"}\SpecialCharTok{\textbackslash{}n}\StringTok{"}\NormalTok{)}
\end{Highlighting}
\end{Shaded}

\begin{verbatim}
## Number of female observations:  14
\end{verbatim}

\begin{Shaded}
\begin{Highlighting}[]
\FunctionTok{cat}\NormalTok{(}\StringTok{"Number of male obserbations: "}\NormalTok{,  num\_male, }\StringTok{"}\SpecialCharTok{\textbackslash{}n}\StringTok{"}\NormalTok{)}
\end{Highlighting}
\end{Shaded}

\begin{verbatim}
## Number of male obserbations:  14
\end{verbatim}

\begin{Shaded}
\begin{Highlighting}[]
\CommentTok{\#c. Creating a Bar Plot for the Number of Males and Females}
\NormalTok{gender\_counts }\OtherTok{\textless{}{-}} \FunctionTok{table}\NormalTok{(household\_data}\SpecialCharTok{$}\NormalTok{Gender)}

\FunctionTok{barplot}\NormalTok{(gender\_counts,}
        \AttributeTok{main =} \StringTok{"Number of Males and Females in Household Data"}\NormalTok{,}
        \AttributeTok{xlab =} \StringTok{"Gender"}\NormalTok{,}
        \AttributeTok{ylab =} \StringTok{"Count"}\NormalTok{,}
        \AttributeTok{col =} \FunctionTok{c}\NormalTok{(}\StringTok{"lightpink"}\NormalTok{, }\StringTok{"lightblue"}\NormalTok{),}
        \AttributeTok{legend =} \FunctionTok{c}\NormalTok{(}\StringTok{"Female"}\NormalTok{, }\StringTok{"Male"}\NormalTok{))}
\end{Highlighting}
\end{Shaded}

\includegraphics{RWorksheet-Barrientos4B_files/figure-latex/unnamed-chunk-2-1.pdf}

\begin{Shaded}
\begin{Highlighting}[]
\CommentTok{\# 5. }
\CommentTok{\# a. Create a piechart that will include labels in percentage.Add some colors and title of the chart. Write the R scripts and show its output.}

\NormalTok{expenses }\OtherTok{\textless{}{-}} \FunctionTok{c}\NormalTok{(}\AttributeTok{Food =} \DecValTok{60}\NormalTok{, }\AttributeTok{Electricity =} \DecValTok{10}\NormalTok{, }\AttributeTok{Savings =} \DecValTok{5}\NormalTok{, }\AttributeTok{Miscellaneous =} \DecValTok{25}\NormalTok{)}

\NormalTok{percent\_expenses }\OtherTok{\textless{}{-}} \FunctionTok{round}\NormalTok{(}\DecValTok{100} \SpecialCharTok{*}\NormalTok{ expenses }\SpecialCharTok{/} \FunctionTok{sum}\NormalTok{(expenses))}
\NormalTok{labels }\OtherTok{\textless{}{-}} \FunctionTok{paste}\NormalTok{(}\FunctionTok{names}\NormalTok{(percent\_expenses), percent\_expenses, }\StringTok{"\%"}\NormalTok{)}
\NormalTok{colors }\OtherTok{\textless{}{-}} \FunctionTok{c}\NormalTok{(}\StringTok{"violet"}\NormalTok{, }\StringTok{"orange"}\NormalTok{, }\StringTok{"lightgreen"}\NormalTok{, }\StringTok{"pink"}\NormalTok{)}

\FunctionTok{pie}\NormalTok{(expenses, }\AttributeTok{labels =}\NormalTok{ labels, }\AttributeTok{col =}\NormalTok{ colors, }\AttributeTok{main =} \StringTok{"Dela Cruz Family Monthly Expenses"}\NormalTok{)}
\end{Highlighting}
\end{Shaded}

\includegraphics{RWorksheet-Barrientos4B_files/figure-latex/unnamed-chunk-3-1.pdf}

\begin{Shaded}
\begin{Highlighting}[]
\CommentTok{\# 6. Use the iris dataset.}
\CommentTok{\#a. Check for the structure of the dataset using the str() function. Describe what you have seen in the output.}
\FunctionTok{str}\NormalTok{(iris)}
\end{Highlighting}
\end{Shaded}

\begin{verbatim}
## 'data.frame':    150 obs. of  5 variables:
##  $ Sepal.Length: num  5.1 4.9 4.7 4.6 5 5.4 4.6 5 4.4 4.9 ...
##  $ Sepal.Width : num  3.5 3 3.2 3.1 3.6 3.9 3.4 3.4 2.9 3.1 ...
##  $ Petal.Length: num  1.4 1.4 1.3 1.5 1.4 1.7 1.4 1.5 1.4 1.5 ...
##  $ Petal.Width : num  0.2 0.2 0.2 0.2 0.2 0.4 0.3 0.2 0.2 0.1 ...
##  $ Species     : Factor w/ 3 levels "setosa","versicolor",..: 1 1 1 1 1 1 1 1 1 1 ...
\end{verbatim}

\begin{Shaded}
\begin{Highlighting}[]
\CommentTok{\#b. Create an R object that will contain the mean of the sepal.length, sepal.width,petal.length,and petal.width. What is the R script and its result?}
\NormalTok{mean\_values }\OtherTok{\textless{}{-}} \FunctionTok{c}\NormalTok{(}\FunctionTok{mean}\NormalTok{(iris}\SpecialCharTok{$}\NormalTok{Sepal.Length), }\FunctionTok{mean}\NormalTok{(iris}\SpecialCharTok{$}\NormalTok{Sepal.Width), }\FunctionTok{mean}\NormalTok{(iris}\SpecialCharTok{$}\NormalTok{Petal.Length), }\FunctionTok{mean}\NormalTok{(iris}\SpecialCharTok{$}\NormalTok{Petal.Width))}
\NormalTok{mean\_values}
\end{Highlighting}
\end{Shaded}

\begin{verbatim}
## [1] 5.843333 3.057333 3.758000 1.199333
\end{verbatim}

\begin{Shaded}
\begin{Highlighting}[]
\CommentTok{\#c. Create a pie chart for the Species distribution. Add title, legends, and colors. Write the R script and its result.}
\NormalTok{species\_counts }\OtherTok{\textless{}{-}} \FunctionTok{table}\NormalTok{(iris}\SpecialCharTok{$}\NormalTok{Species)}
\NormalTok{colors }\OtherTok{\textless{}{-}} \FunctionTok{c}\NormalTok{(}\StringTok{"violet"}\NormalTok{, }\StringTok{"orange"}\NormalTok{, }\StringTok{"lightgreen"}\NormalTok{)}
\FunctionTok{pie}\NormalTok{(species\_counts, }\AttributeTok{labels =}\NormalTok{ labels, }\AttributeTok{col =}\NormalTok{ colors, }\AttributeTok{main =} \StringTok{"Iris Species Distribution"}\NormalTok{)}
\end{Highlighting}
\end{Shaded}

\includegraphics{RWorksheet-Barrientos4B_files/figure-latex/unnamed-chunk-3-2.pdf}

\begin{Shaded}
\begin{Highlighting}[]
\CommentTok{\#d. Subset the species into setosa, versicolor, and virginica. Write the R scripts and show the last six (6) rows of each species.}
\NormalTok{setosa }\OtherTok{\textless{}{-}} \FunctionTok{subset}\NormalTok{(iris, Species }\SpecialCharTok{==} \StringTok{"setosa"}\NormalTok{)}
\NormalTok{versicolor }\OtherTok{\textless{}{-}} \FunctionTok{subset}\NormalTok{(iris, Species }\SpecialCharTok{==} \StringTok{"versicolor"}\NormalTok{)}
\NormalTok{virginica }\OtherTok{\textless{}{-}} \FunctionTok{subset}\NormalTok{(iris, Species }\SpecialCharTok{==} \StringTok{"virginica"}\NormalTok{)}

\FunctionTok{tail}\NormalTok{(setosa)}
\end{Highlighting}
\end{Shaded}

\begin{verbatim}
##    Sepal.Length Sepal.Width Petal.Length Petal.Width Species
## 45          5.1         3.8          1.9         0.4  setosa
## 46          4.8         3.0          1.4         0.3  setosa
## 47          5.1         3.8          1.6         0.2  setosa
## 48          4.6         3.2          1.4         0.2  setosa
## 49          5.3         3.7          1.5         0.2  setosa
## 50          5.0         3.3          1.4         0.2  setosa
\end{verbatim}

\begin{Shaded}
\begin{Highlighting}[]
\CommentTok{\#e. Create a scatterplot of the sepal.length and sepal.width using the different species(setosa,versicolor,virginica). }
\CommentTok{\#Add a title = “Iris Dataset”, subtitle = “Sepal}
\CommentTok{\#width and length, labels for the x and y axis, the pch symbol and colors should be based on the species.}

\CommentTok{\#Hint: Need to convert to factors the species to store categorical variables.}
\FunctionTok{as.factor}\NormalTok{(iris}\SpecialCharTok{$}\NormalTok{Species)}
\end{Highlighting}
\end{Shaded}

\begin{verbatim}
##   [1] setosa     setosa     setosa     setosa     setosa     setosa    
##   [7] setosa     setosa     setosa     setosa     setosa     setosa    
##  [13] setosa     setosa     setosa     setosa     setosa     setosa    
##  [19] setosa     setosa     setosa     setosa     setosa     setosa    
##  [25] setosa     setosa     setosa     setosa     setosa     setosa    
##  [31] setosa     setosa     setosa     setosa     setosa     setosa    
##  [37] setosa     setosa     setosa     setosa     setosa     setosa    
##  [43] setosa     setosa     setosa     setosa     setosa     setosa    
##  [49] setosa     setosa     versicolor versicolor versicolor versicolor
##  [55] versicolor versicolor versicolor versicolor versicolor versicolor
##  [61] versicolor versicolor versicolor versicolor versicolor versicolor
##  [67] versicolor versicolor versicolor versicolor versicolor versicolor
##  [73] versicolor versicolor versicolor versicolor versicolor versicolor
##  [79] versicolor versicolor versicolor versicolor versicolor versicolor
##  [85] versicolor versicolor versicolor versicolor versicolor versicolor
##  [91] versicolor versicolor versicolor versicolor versicolor versicolor
##  [97] versicolor versicolor versicolor versicolor virginica  virginica 
## [103] virginica  virginica  virginica  virginica  virginica  virginica 
## [109] virginica  virginica  virginica  virginica  virginica  virginica 
## [115] virginica  virginica  virginica  virginica  virginica  virginica 
## [121] virginica  virginica  virginica  virginica  virginica  virginica 
## [127] virginica  virginica  virginica  virginica  virginica  virginica 
## [133] virginica  virginica  virginica  virginica  virginica  virginica 
## [139] virginica  virginica  virginica  virginica  virginica  virginica 
## [145] virginica  virginica  virginica  virginica  virginica  virginica 
## Levels: setosa versicolor virginica
\end{verbatim}

\begin{Shaded}
\begin{Highlighting}[]
\FunctionTok{plot}\NormalTok{(setosa}\SpecialCharTok{$}\NormalTok{Sepal.Length, setosa}\SpecialCharTok{$}\NormalTok{Sepal.Width, }\AttributeTok{pch =} \DecValTok{19}\NormalTok{, }\AttributeTok{col =} \StringTok{"red"}\NormalTok{, }\AttributeTok{xlab =} \StringTok{"Sepal Length"}\NormalTok{, }\AttributeTok{ylab =} \StringTok{"Sepal Width"}\NormalTok{, }\AttributeTok{main =} \StringTok{"Iris Dataset"}\NormalTok{, }\AttributeTok{sub =} \StringTok{"Sepal width and length"}\NormalTok{)}
\FunctionTok{points}\NormalTok{(versicolor}\SpecialCharTok{$}\NormalTok{Sepal.Length, versicolor}\SpecialCharTok{$}\NormalTok{Sepal.Width, }\AttributeTok{pch =} \DecValTok{19}\NormalTok{, }\AttributeTok{col =} \StringTok{"blue"}\NormalTok{)}
\FunctionTok{points}\NormalTok{(virginica}\SpecialCharTok{$}\NormalTok{Sepal.Length, virginica}\SpecialCharTok{$}\NormalTok{Sepal.Width, }\AttributeTok{pch =} \DecValTok{19}\NormalTok{, }\AttributeTok{col =} \StringTok{"green"}\NormalTok{)}
\FunctionTok{legend}\NormalTok{(}\StringTok{"topright"}\NormalTok{, }\AttributeTok{legend =} \FunctionTok{levels}\NormalTok{(iris}\SpecialCharTok{$}\NormalTok{Species), }\AttributeTok{col =} \FunctionTok{c}\NormalTok{(}\StringTok{"red"}\NormalTok{, }\StringTok{"blue"}\NormalTok{, }\StringTok{"green"}\NormalTok{), }\AttributeTok{pch =} \DecValTok{18}\NormalTok{)}
\end{Highlighting}
\end{Shaded}

\includegraphics{RWorksheet-Barrientos4B_files/figure-latex/unnamed-chunk-3-3.pdf}

\begin{Shaded}
\begin{Highlighting}[]
\CommentTok{\#f. Interpret the result.}
\CommentTok{\#The scatterplot shows the relationship between the sepal length and sepal width of the iris dataset. }
\end{Highlighting}
\end{Shaded}

\section{Basic Cleaning and Transformation of
Objects}\label{basic-cleaning-and-transformation-of-objects}

\begin{Shaded}
\begin{Highlighting}[]
\CommentTok{\#7. Import the alexa{-}file.xlsx. Check on the variations. Notice that there are ex{-}tra whitespaces among black variants (Black Dot, Black Plus, Black Show, BlackSpot). Also on the white variants (White Dot, White Plus, White Show, White Spot).}
\FunctionTok{library}\NormalTok{(readxl)}
\NormalTok{alexa\_data }\OtherTok{\textless{}{-}} \FunctionTok{read\_excel}\NormalTok{(}\StringTok{"C:/PROJ/alexa\_file.xlsx"}\NormalTok{)}
\NormalTok{alexa\_data}
\end{Highlighting}
\end{Shaded}

\begin{verbatim}
## # A tibble: 3,150 x 5
##    rating date                variation           verified_reviews      feedback
##     <dbl> <dttm>              <chr>               <chr>                    <dbl>
##  1      5 2018-07-31 00:00:00 Charcoal Fabric     Love my Echo!                1
##  2      5 2018-07-31 00:00:00 Charcoal Fabric     Loved it!                    1
##  3      4 2018-07-31 00:00:00 Walnut Finish       Sometimes while play~        1
##  4      5 2018-07-31 00:00:00 Charcoal Fabric     I have had a lot of ~        1
##  5      5 2018-07-31 00:00:00 Charcoal Fabric     Music                        1
##  6      5 2018-07-31 00:00:00 Heather Gray Fabric I received the echo ~        1
##  7      3 2018-07-31 00:00:00 Sandstone Fabric    Without having a cel~        1
##  8      5 2018-07-31 00:00:00 Charcoal Fabric     I think this is the ~        1
##  9      5 2018-07-30 00:00:00 Heather Gray Fabric looks great                  1
## 10      5 2018-07-30 00:00:00 Heather Gray Fabric Love it! I’ve listen~        1
## # i 3,140 more rows
\end{verbatim}

\begin{Shaded}
\begin{Highlighting}[]
\CommentTok{\#a. Rename the white and black variants by using gsub() function.}
\NormalTok{alexa\_data}\SpecialCharTok{$}\NormalTok{variation }\OtherTok{\textless{}{-}} \FunctionTok{gsub}\NormalTok{(}\StringTok{"Black Dot"}\NormalTok{, }\StringTok{"BlackDot"}\NormalTok{, alexa\_data}\SpecialCharTok{$}\NormalTok{variation)}
\NormalTok{alexa\_data}\SpecialCharTok{$}\NormalTok{variation }\OtherTok{\textless{}{-}} \FunctionTok{gsub}\NormalTok{(}\StringTok{"Black Plus"}\NormalTok{, }\StringTok{"BlackPlus"}\NormalTok{, alexa\_data}\SpecialCharTok{$}\NormalTok{variation)}
\NormalTok{alexa\_data}\SpecialCharTok{$}\NormalTok{variation }\OtherTok{\textless{}{-}} \FunctionTok{gsub}\NormalTok{(}\StringTok{"Black Show"}\NormalTok{, }\StringTok{"BlackShow"}\NormalTok{, alexa\_data}\SpecialCharTok{$}\NormalTok{variation)}
\NormalTok{alexa\_data}\SpecialCharTok{$}\NormalTok{variation }\OtherTok{\textless{}{-}} \FunctionTok{gsub}\NormalTok{(}\StringTok{"Black Spot"}\NormalTok{, }\StringTok{"BlackSpot"}\NormalTok{, alexa\_data}\SpecialCharTok{$}\NormalTok{variation)}
\NormalTok{alexa\_data}\SpecialCharTok{$}\NormalTok{variation }\OtherTok{\textless{}{-}} \FunctionTok{gsub}\NormalTok{(}\StringTok{"White Dot"}\NormalTok{, }\StringTok{"WhiteDot"}\NormalTok{, alexa\_data}\SpecialCharTok{$}\NormalTok{variation)}
\NormalTok{alexa\_data}\SpecialCharTok{$}\NormalTok{variation }\OtherTok{\textless{}{-}} \FunctionTok{gsub}\NormalTok{(}\StringTok{"White Plus"}\NormalTok{, }\StringTok{"WhitePlus"}\NormalTok{, alexa\_data}\SpecialCharTok{$}\NormalTok{variation)}
\NormalTok{alexa\_data}\SpecialCharTok{$}\NormalTok{variation }\OtherTok{\textless{}{-}} \FunctionTok{gsub}\NormalTok{(}\StringTok{"White Show"}\NormalTok{, }\StringTok{"WhiteShow"}\NormalTok{, alexa\_data}\SpecialCharTok{$}\NormalTok{variation)}
\NormalTok{alexa\_data}\SpecialCharTok{$}\NormalTok{variation }\OtherTok{\textless{}{-}} \FunctionTok{gsub}\NormalTok{(}\StringTok{"White Spot"}\NormalTok{, }\StringTok{"WhiteSpot"}\NormalTok{, alexa\_data}\SpecialCharTok{$}\NormalTok{variation)}

\FunctionTok{head}\NormalTok{(alexa\_data)}
\end{Highlighting}
\end{Shaded}

\begin{verbatim}
## # A tibble: 6 x 5
##   rating date                variation           verified_reviews       feedback
##    <dbl> <dttm>              <chr>               <chr>                     <dbl>
## 1      5 2018-07-31 00:00:00 Charcoal Fabric     Love my Echo!                 1
## 2      5 2018-07-31 00:00:00 Charcoal Fabric     Loved it!                     1
## 3      4 2018-07-31 00:00:00 Walnut Finish       Sometimes while playi~        1
## 4      5 2018-07-31 00:00:00 Charcoal Fabric     I have had a lot of f~        1
## 5      5 2018-07-31 00:00:00 Charcoal Fabric     Music                         1
## 6      5 2018-07-31 00:00:00 Heather Gray Fabric I received the echo a~        1
\end{verbatim}

\begin{Shaded}
\begin{Highlighting}[]
\CommentTok{\#Write the R scripts and show an example of the output by getting a snippet. To embed an image into Rmd, use the function below:}
\NormalTok{knitr}\SpecialCharTok{::}\FunctionTok{include\_graphics}\NormalTok{(}\StringTok{"C:/PROJ/alexa\_file.xlsx"}\NormalTok{)}
\end{Highlighting}
\end{Shaded}

\includegraphics{../../../alexa_file.xlsx}

\end{document}
